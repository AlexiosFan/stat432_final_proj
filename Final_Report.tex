\documentclass[8pt]{article}
\usepackage{amssymb}
\usepackage{babel}
\usepackage{geometry}
\usepackage{amsmath}
\usepackage{amsthm}
\usepackage{framed}
\usepackage{pifont}
\usepackage{listings}
\usepackage{tikz}
\usepackage{hyperref}
\usepackage{graphicx}
\usepackage{float}
\usepackage{paralist}
\usepackage{xcolor}
\usepackage{tikz}

\usetikzlibrary{automata, positioning, arrows}


\def\ra{\rightarrow}
\def\rr{\Rightarrow}
\def\tf{$\Rightarrow$}
\def\oo{\infty}
\def\l/{\backslash}
\def\0{\emptyset}
\def\P{\mathbb{P}}
\def\px{\mathcal{P}_X}
\def\s{\mathcal{S}}
\def\a{\mathcal{A}}
\def\bs{\mathcal{B}}
\def\lm{\mathcal{L}}
\def\R{\mathbb{R}}
\def\E{\mathbb{E}}
\def\Z{\mathbb{Z}}
\def\m{\mathcal{M}}
\def\N{\mathcal{N}}
\def\b{\,\,\,}
\def\activity{\textit{activity} }
\def\down{\textit{down\_event} }
\def\up{\textit{up\_event} }
\def\textchange{\textit{text\_change} }
\def\cursor{\textit{cursor\_position} }
\def\wordcount{\textit{word\_count} }
\def\id{\textit{id} }
\def\eventid{\textit{event\_id} }
\def\downtime{\textit{down\_time} }
\def\uptime{\textit{up\_time} }
\def\actiontime{\textit{action\_time} }
\def\Input{\textit{Input} }
\def\Nonproduction{\textit{Nonproduction} }
\def\Remove{\textit{Remove/Cut} }
\def\Paste{\textit{Paste} }
\def\Replace{\textit{Replace} }
\def\MoveFromTo{\textit{Move From To} }



\title{STAT 432 Final Project}
\author{Ruiming Min(rmin4), Zixuan Fan(zixuanf5), Yewen Li(yewenli2)}
\date{\today}

\begin{document}

\maketitle

\section{Project Description and Summary}

\section{Literature Review}
On the Kaggle website, there are many discussions with profound quality and insightful materials. 
In general, participants tend to use the modern deep learning models such as convolution neural network,
transformer model etc. The classical statistical learning models do not prove to be effective in this competition. 
Potential reasons for this phenomenon are that the classical models are not able to capture the complex patterns in the data. 
The writing style of the text does not follow any explicit linear or quadratic patterns. 
For this reason, the regression models, which is based on the linear assumption, are not able to capture the patterns in the data.
On the other hand, the classification models, also makes assumptions on the distribution of data in certain clusters, 
while the distribution of the handwriting data is highly dependent on our data processing. 
Thus, the classification models may not be effective, either, because some patterns are either lost in processing or cannot be summarized into single 
record for each observation. 

Looking into various discussions, we found two revealed methods highly effective with detailed descriptions. 
We analyze those methods and exploit some features in our project. 
\subsection{LGBM Regression}
\href{https://www.kaggle.com/code/hengzheng/link-writing-simple-lgbm-baseline}{Link Writing Simple LGBM baseline} by ZhengHeng
presented the author's Python implementation of LGBM regression model for score prediction. The author's approach 
reached a score of 0.589 and ranked 382nd on the leaderboard. Although the author was not amongst the top of the leaderboard, 
his approach was simple and inspired many other people, as shown in this \href{https://www.kaggle.com/competitions/linking-writing-processes-to-writing-quality/discussion/451081}{discussion}.

ZhengHeng's approach can be summarized into two steps. First, he did a feature engineering on the data. 
By considering the physical meaning of the relationship between features, he generated some new features from 
the existing log and discarded those high correlated features. Some of new features are the ratio of 
\textbf{word count}, \textbf{idle time} and \textbf{event time}. Second, he trained the LGBM regression model
on a five-fold cross validation. The Light Gradient Boosting Machine (LGBM) is a gradient boosting framework
that uses tree based learning algorithms. 
This newly developed model grows tree vertically in a leaf-wise fashion, while the traditional tree-based models
grow horizontally in a level-wise fashion. The author tuned the model manually and publish the best parameters. 
% Decide if we want to use this regressor?

\subsection{Feature Engineering}
\href{https://www.kaggle.com/code/hiarsl/feature-engineering-sentence-paragraph-features}{Feature Engineering Sentence and Pargraph Feature} by Matthias Hauser
gives us a detailed description of feature engineering based on the nature of the training dataset and also the mean importance of its variables. Based on the author's feature engineering,
the author improved his score to 0.586 and ranked 54th on the leaderboard. Also, this post got a gold medal in the notebook competition. As one of the highest-ranked notebooks, 
his work is very valuable and worth learning from. There are also many discussions about this notebook, which is shown in this \href{https://www.kaggle.com/code/hiarsl/feature-engineering-sentence-paragraph-features/comments}{discussion}.

Matthias Hauser's work can be summarized into three groups: \textbf{sentence features, paragraph features and other features}. 
For sentence features, the author used the average length of sentences, the number of sentences, and so on. 
For paragraph features, the author used the average length of paragraphs, the length of the first paragraph, the number of paragraphs and so on.
For other features, the author used the average length of words, the number of words, pause time, and so on.
These features are very useful and meaningful which could be used in our project. Moreover, the author's work gives us a good implementation of feature engineering.

Futhermore, the author also gave us the mean importance of his variables. The graph is shown at the end of \href{https://www.kaggle.com/code/hiarsl/feature-engineering-sentence-paragraph-features/notebook}{his post}.
The importance of his variables shows that the importance of his variables is very close to our natural knowledge. Some of the most important variables are \textbf{the length of words}, \textbf{pause time}, and \textbf{sentence length}, which are also the standard by which we judge a person’s writing level in life.
Based on his work, we could do more feature engineering like delete ratio, text change ratio, and paragraph balance, to improve our score.

\section{Data processing}
Based on \href{https://www.kaggle.com/competitions/linking-writing-processes-to-writing-quality/data}{the structure of the data on Kagge}, each indevidual is represented by a sequence of events and each event has plenty of variables.
The variables include the following:
\begin{compactitem}
    \item \textbf{\textit{id}} - The unique ID of the essay.
    \item \textbf{\textit{event\_id}} - The index of the event, ordered chronologically.
    \item \textbf{\textit{down\_time}} - The time of the down event in milliseconds.
    \item \textbf{\textit{up\_time}} - The time of the up event in milliseconds.
    \item \textbf{\textit{action\_time}} - The duration of the event (the difference between down\_time and up\_time).
    \item \textbf{\textit{activity}} - The category of activity which the event belongs to.
    \begin{compactitem}
        \item \textit{Nonproduction} - The event does not alter the text in any way.
        \item \textit{Input} - The event adds text to the essay.
        \item \textit{Remove/Cut} - The event removes text from the essay.
        \item \textit{Paste} - The event changes the text through a paste input.
        \item \textit{Replace} - The event replaces a section of text with another string.
        \item \textit{Move From} \([x1, y1]\) To \([x2, y2]\) - The event moves a section of text spanning character index \(x1, y1\) to a new location \(x2, y2\).
    \end{compactitem}
    \item \textbf{\textit{down\_event}} - The name of the event when the key/mouse is pressed.
    \item \textbf{\textit{up\_event}} - The name of the event when the key/mouse is released.
    \item \textbf{\textit{text\_change}} - The text that changed as a result of the event (if any).
    \item \textbf{\textit{cursor\_position}} - The character index of the text cursor after the event.
    \item \textbf{\textit{word\_count}} - The word count of the essay after the event.
\end{compactitem}
This would make the data very sparse and hard to process.
Especially, those non-numeric variables like \activity are hard to directly use in the model. 
Therefore, in this project, effective data processing will be a crucial component and will significantly impact future outcomes.
To achieve better data processing results, the data processing in this project can be broadly divided into three parts: \textbf{data integration}, \textbf{feature engineering}, and \textbf{dimensionality reduction}.

\subsection{Data Integration}
In order to integrate the data, getting the whole text of each individual's essay is the first step.
Based on the data structure, follwoing the \eventid and the instructions of \activity, we designed a algorithm to get the whole text of each individual's essay.

Also, the complicated construction of the \activity makes the directly use of it impossible. To summarize and integrate it, we designed a algorithm to get the \textbf{total number} of each \activity.
Besides, from the basic understanding of the quality of writing, the writing style of the text also plays an important role in the quality of writing.
Therefore, we designed a algorithm to get the whole number of each indeviduals' \textbf{total number} of change their activity.

After dealing with the \activity and its related variables (\down , \up , \textchange , \cursor , and \wordcount), we also need to deal with \downtime , \uptime , and \actiontime . 
For \actiontime , there are three obviously essential indexs: \textbf{mean}, \textbf{variance}, and \textbf{sum}.
However, since downtime and \uptime contain the informatin of the time point insteade of a interval, the situation is a little bit different. 
According to the concept of writing skill, thinking time is an important variable.
Therefore, we designed a algorithm to get the sequnece of \textit{pause\_time}, which is the difference between the \up of the previous event and the \down of the next event.
Then, same as the \actiontime , we get the \textbf{mean}, \textbf{variance}, and \textbf{sum} of the \textit{pause\_time}.

To summarize the data integration, we get the following variables:
\begin{compactitem}
    \item \textbf{\textit{id}} - The unique ID of the essay.
    \item \textbf{\text{essay\_text}} - The whole text of the essay.
    \item \textbf{\textit{num\_nonproduction}} - The total number of \textit{Nonproduction} activity.
    \item \textbf{\textit{num\_input}} - The total number of \textit{Input} activity.
    \item \textbf{\textit{num\_remove}} - The total number of \textit{Remove/Cut} activity.
    \item \textbf{\textit{num\_paste}} - The total number of \textit{Paste} activity.
    \item \textbf{\textit{num\_replace}} - The total number of \textit{Replace} activity.
    \item \textbf{\textit{num\_move}} - The total number of \textit{Move From To} activity.
    \item \textbf{\textit{num\_change\_activity}} - The total number of change activity.
    \item \textbf{\textit{mean\_action\_time}} - The mean of \textit{action\_time}.
    \item \textbf{\textit{var\_action\_time}} - The variance of \textit{action\_time}.
    \item \textbf{\textit{sum\_action\_time}} - The sum of \textit{action\_time}.
    \item \textbf{\textit{mean\_pause\_time}} - The mean of \textit{pause\_time}.
    \item \textbf{\textit{var\_pause\_time}} - The variance of \textit{pause\_time}.
    \item \textbf{\textit{sum\_pause\_time}} - The sum of \textit{pause\_time}.
    \item \textbf{\textit{score}} - The score of the essay.
\end{compactitem}
which can be expressed as a vector for each individual.

\subsection{Feature Engineering}
From the literature review above, it is not difficult to see that the general criteria for judging the quality of writing are still applicable when assessing the merits through machine learning.

\subsection{Dimensionality Reduction}


\section{Unsupervised Learning}

\subsection{* model1}


\subsection{* model2}


\section{Regression/Classification Models}

\subsection{Regression Model 1: Linear Regression with ridge/lasso penalty}
Since we have learnt from the Kaggle discussion that the linear regression models are not really effective for this dataset, 
we consider a penalty and see if it leads to a good performance. 
We start with raw Ridge and LASSO regression models. Using the \texttt{glmnet} package with 10-fold cross validation, 
we get the $R^2 \approx 0.34$ for both Ridge and LASSO regression when predicting with \texttt{lambda.min}. 
This is not a good result, so we try the elastic net and lambda values for tuning. 
\paragraph{Elastic Net}
The elastic net is a combination of the Ridge and LASSO regression as shown in the lecture. 
The related tuning parameter is $\alpha$, which decides the proportion of the two penalties. 
In this setting, we iterate all $\alpha \in [0, 1]$ with 0.01 increments. 
In additional, we use the default lambda sequences for each $\alpha$. 
The tuning result is shown in \hyperref[fig:elastic_net]{Figure 2}.
\begin{figure}[H]
    \centering
    \includegraphics*[scale=0.25]{figures/elastic_net.png}
    \caption{Elastic Net: Tuning $\alpha$}
    \label{fig:elastic_net}
\end{figure}
It is clear that the LASSO penalty $\alpha = 1$ does not really help with the prediction.
The best $R^2$ is reached with the pure Ridge model. However, the difference is not significant, 
because of the fact that the linear assumption may not hold for this dataset.
\paragraph{Tuning Lambda}
The default lambda sequence may not be the best choice for our dataset.
Thus, we use a grid of lambda with exponential growth to find 
if a large penalty can influence the prediction. The lambda grid we use is 
\begin{align*}
    \texttt{lambda} = \texttt{exp(seq(-5, 5, 0.05))}
\end{align*}
The average tuning result with 100 simulations is shown in \hyperref[fig:lambda]{Figure 3}.
\begin{figure}[H]
    \centering
    \includegraphics*[scale=0.25]{figures/lambda.png}
    \caption{Ridge: Tuning $\lambda$}
\label{fig:lambda}
\end{figure}
Unexpectedly, the best $R^2$ is reached with the smallest lambda. 
It indicates that the Ridge penalty cannot help with the prediction, either. 
The interpretation of the bad fitting of linear regression model still lies in the absence of linear patterns in the data.
Even though Ridge penalty helps to resolve high multicollinearity and LASSO penalty 
helps to select features, they cannot help with the prediction without the linear assumption.

\subsection{Regression Model 2: *}

\subsection{Regression Model 3: LightBGM}
This model is inspired by ZhengHeng's method as presented in the literature review.
To recap the method, it 

\subsection{Classification Model: Random Forest}
For classification, we choose the random forest model using the R package \texttt{randomForest}.
The simple fitting using the default parameters gives us an \texttt{accuracy} of $\alpha = 0.2848$. 
Taking look at the confusion matrix, we found that prediction is concentrated along the diagonal, 
but does not have a high accuracy. So we present a new metric, \texttt{accuracy with tolerance}. 
This is the percentage of observations that are predicted within a tolerance $0.5$ of the true value.
\begin{align*}
    \tau = \texttt{Accuracy with tolerance} = \dfrac{\sum \texttt{diag} + \sum \texttt{subdiag} + \sum \texttt{superdiag}}{\sum \texttt{Observations}}
\end{align*}
In the simple fitting, $\tau = 0.6909$, which proves a good concentrtion of random forest model. 
\paragraph{Parameter Tuning}
The tuning goal of the model is to increase both the \texttt{accuracy} and the \texttt{accuracy with tolerance}.
Amongst many tuning parameters of random forest, we focus on \texttt{mtry} and \texttt{nodesize} with default 
\texttt{ntree} = 500. The grid we used is 
\begin{align*}
    \texttt{mtry} &\in \{1, 3, 5, 8, 10, 12, 15, 20, 25, 40,  50 \} \\
    \texttt{nodesize} &\in \{1, 3, 5, 8, 10, 12, 15, 20, 25, 40,  50 \}
\end{align*}
where the default values are $\texttt{mtry} = 3$ and $\texttt{nodesize} = 1$. 
The \hyperref[fig:contour]{contour plots} shows a best fit at $\texttt{mtry} \approx 3$ and $\texttt{nodesize} \approx 10$.
We may conclude that the effect of the random forest model is limited at $\alpha \approx 0.3$
and $\tau \approx 0.7$. Since there is an obvious concentration, 
this model is useful in the sense that it can show a general trend of the score.
However, it is not a good choice for sophisticated prediction as of the low accuracy.
\begin{figure}[H]
    \includegraphics*[scale=0.25]{figures/contour_plot_accuracy.png}
    \includegraphics*[scale=0.25]{figures/contour_plot_accuracy_with_tolerance.png}
    \caption{Contour plots of \texttt{accuracy} and \texttt{accuracy with tolerance}}
    \label{fig:contour}
\end{figure}

\paragraph{Variable Importance}
Another property of the random forest model is that it can provide a report on variable importance. 
Taking a look at \texttt{MeanDecreaseAccuracy} and \texttt{MeanDecreaseGini}, 
we found that the ratios features are much more importance than the raw value features. 
For instance, the \texttt{MeanDecreaseAccuracy} of \texttt{ratio\_word\_count} is $81.40$, 
while that of \texttt{Input\_count} is only $5.94$. We may tell from this phenomenon that
the ratio features are more informative than the raw value features. 
And a good feature engineering strategy results in better classification accuracy. 

\end{document}